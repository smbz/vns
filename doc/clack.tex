\documentclass[a4paper,12pt]{article}

\begin{document}

\title{Using the Clack Graphical Router with VNS}
\date{4th August 2011}

\section{Creating Topology Templates for Clack}
Clack is fairly specific about topology of the network it is given:
\begin{itemize}
\item The gateway node must be named \texttt{Gateway}.  This will appear as a
firewall, and must have just one interface.
\item Although Clack shows an ``Internet'' node on the screen, this has no
counterpart within VNS, and should not be created.
\end{itemize}

\section{Interaction Between Clack and VNS}
Clack is configured with XML files retrieved from the VNS server.  There are two
of these files: one gives Clack its internal configuration and the other gives
it the topology inside the VNS server.

\subsection{Clack configuration XML}
The first file, the Clack configuration, contains (among other things) the
layout of the network graph on the screen and the connection of components
inside the router part of clack.  Every node that appears in the VNS topology
must have an entry for its location on the screen in the Clack configuration
XML or Clack will refuse to load it.

The Clack configuration file also contains a topology number and the URL of the
VNS configuration file.  The topology number specifies a range of topologies,
e.g. ``10-20''.  The VNS configuration URL has, for recent versions of VNS, an
authentication token which is submitted by HTTP GET to ensure that the user is
permitted to access the VNS configuration for the specified topology.

The Clack configuration XML also contains the username and authentication token
of the user to allow them to log on to VNS.

\subsection{VNS configuration XML}
The second file specifies the topology layout as understood by VNS.  Clack uses
this to create default routing tables, show information about each node, etc.
Note that the address of the gateway is specified in the Clack configuration
file in \texttt{net/clackrouter/jgraph/pad/resources/Clack.properties} under
\texttt{FIREWALL\_ADDRESS}, and this setting takes priority over the XML sent by
the VNS web server.  The \texttt{VNS\_SERVER\_ADDRESS} property in the same file
specifies the default VNS server address, and although it is not necessary to
change this it is recommended.

\subsection{Building Clack}
The Clack source code is (at time of writing) hosted at
\texttt{http://yuba.stanford.edu/vns/clack/} and
\texttt{http://gitorious.org/clack-graphical-router}.

Before building Clack, you will most likely want to change the default server
and gateway as described above.  If you intend to run Clack as an applet, you
will also need to create a keystore and put a key in it; this can be done by
running \texttt{\$ keytool -genkey -alias clack -keystore clackkeystore} in the
Clack root directory.  The password for the keystore should be ``stanford''.

To actually build Clack, run \texttt{\$ ant dist}; this will create a dist
directory containing (among other things) several zip files, one of which
contains \texttt{clack-\$\{VERSION\}.jar}.  This is the jar file which needs to
be linked to from the applet.

The HTML for linking to the applet is something like this:

\begin{verbatim}
<applet code="net.clackrouter.gui.ClackLoader.class"
        archive="clack-1.7.2.jar">
  <param name="parameters"
         value="-u http://vns.server/topology123/xml/" />
</applet>
\end{verbatim}

\end{document}
